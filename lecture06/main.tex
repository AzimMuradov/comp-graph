\section{Графика: растровая и векторная}
\begin{center}
    Конспект составил: \textit{Даниил Кадочников}
\end{center}

\subsection{Растровая графика}
\textbf{Описание:}
Представляет изображение в виде сетки пикселей, где каждый пиксель имеет свой цвет.

\textbf{Плюсы:}
\begin{itemize}
    \item Легко отображается на любых устройствах;
    \item Проста в изменении через пиксельные редакторы;
    \item Удобна для хранения;
    \item Простое уменьшение за счёт удаления пикселей.
\end{itemize}

\textbf{Минусы:}
\begin{itemize}
    \item Изображение теряет качество при увеличении;
    \item Границы объектов выглядят неровными при масштабировании.
\end{itemize}

\textbf{Особенности:}
\begin{itemize}
    \item Размер файла зависит от разрешения, а не от сложности изображения;
    \item Сложно проецировать на криволинейные поверхности или применять сложные преобразования.
\end{itemize}

\subsection{Векторная графика}
\textbf{Описание:}
Основывается на математическом описании объектов, таких как линии, кривые и полигоны.

\textbf{Плюсы:}
\begin{itemize}
    \item Изображение остаётся гладким независимо от масштаба;
    \item Не теряет качества при увеличении;
    \item Компактно хранится за счёт математических формул.
\end{itemize}

\textbf{Минусы:}
\begin{itemize}
    \item Отображение сложнее, так как требует вычислений;
    \item Изменение может быть сложным, особенно при работе с несколькими элементами.
\end{itemize}

\textbf{Особенности:}
\begin{itemize}
    \item Размер файла зависит от количества и сложности объектов;
    \item Сложнее проектировать для отображения, так как требуется интерпретация формул.
\end{itemize}

\subsection{Цветовые модели: RGB, CMY и HSV}

\subsubsection{RGB (Red, Green, Blue)}
\textbf{Описание:}
Цветовая модель, основанная на аддитивном смешении красного, зелёного и синего света. Используется в цифровых экранах (мониторы, телевизоры, проекторы).

\textbf{Принцип:}
\begin{itemize}
    \item Цвета смешиваются добавлением света. Пересечение всех трёх компонентов образует белый цвет.
    \item Представляется в виде вектора $(R, G, B)$, где каждое значение находится в диапазоне от 0 до 255 включительно.
\end{itemize}

\subsubsection{CMY (Cyan, Magenta, Yellow)}
\textbf{Описание:}
Субтрактивная модель, основанная на вычитании света из белого. Используется в печати.

\textbf{Принцип:}
\begin{itemize}
    \item Цвета вычисляются как дополнения к RGB:
          \[
              C = 1 - R, \quad M = 1 - G, \quad Y = 1 - B.
          \]
    \item Если представить RGB на кругах Эйлера:
          \begin{itemize}
              \item Пересечение $R$ и $G$ даёт Yellow (жёлтый),
              \item Пересечение $R$ и $B$ даёт Magenta (пурпурный),
              \item Пересечение $G$ и $B$ даёт Cyan (голубой),
              \item Пересечение всех трёх даёт белый цвет.
          \end{itemize}
\end{itemize}

\subsubsection{Представление через кватернионы}
Цвет может быть записан как $a + bi + cj + dk$, где:
\begin{itemize}
    \item $a$ — интенсивность, $b, c, d$ — компоненты цвета,
    \item $i, j, k$ — мнимые единицы, соответствующие трем основным цветам, подчиняющиеся следующим правилам:
          \[
              i^2 = j^2 = k^2 = -1, \quad i \cdot j = -k, \quad j \cdot k = -i, \quad k \cdot i = -j.
          \]
\end{itemize}

\subsubsection{HSV (Hue, Saturation, Value)}
\textbf{Описание:}
Модель, объединяющая идеи RGB и CMY для интуитивного восприятия цвета. Представляется в виде конуса с основанием в форме шестиугольника.

\textbf{Принцип:}
\begin{itemize}
    \item Основание — шестиугольник с вершинами $G, C, B, M, R, Y$, где:
          \begin{itemize}
              \item $G$ — зелёный, $C$ — голубой, $B$ — синий, $M$ — пурпурный, $R$ — красный, $Y$ — жёлтый.
          \end{itemize}
    \item Параметры цвета:
          \begin{enumerate}
              \item \textbf{Угол} в основании отвечает за цветовой тон (Hue),
              \item \textbf{Радиус} — насыщенность (Saturation),
              \item \textbf{Высота} конуса — яркость (Value).
          \end{enumerate}
\end{itemize}

\textbf{Применение:}
HSV часто используется в графических редакторах для выбора и настройки цвета.
