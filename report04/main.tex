\report{Сегментация изображений}{Леонид Елкин}

\subsection{Введение}

Сегментация изображений – это процесс разделения изображения на отдельные области или сегменты, которые имеют схожие свойства, такие как цвет, яркость, текстура, контрастность или другие визуальные характеристики. То есть сегментация – это разделение пикселей на классы по какому-то признаку. Объединение данных классов образует все изображение.

Цель состоит в том, чтобы изменить представление изображения на что-то более осмысленное и более простое для анализа. Сегментация обычно используется для определения объектов и границ (линий, кривых и т. д.) на изображениях. Это очень полезно в таких областях, как:

\begin{itemize}
    \item \textbf{Медицина}: Выделение органов или аномалий на медицинских снимках, таких как МРТ или рентген.
    \item \textbf{Автономное вождение}: Определение дорожных разметок, автомобилей, пешеходов и других объектов на дороге.
    \item \textbf{Компьютерное зрение}: Распознавание объектов и отслеживание их движений.
    \item \textbf{Обработка спутниковых снимков}: Классификация областей (вода, растительность, города).
\end{itemize}

Алгоритмы сегментации изображений можно глобально разделить на два больших класса: DL(deep learning) подход и «традиционный» подход. Саму же задачу сегментации можно разделить на три группы:

\begin{itemize}
    \item Семантическая сегментация(semantic segmentation)
    \item Сегментация экземпляров(instance segmentation)
    \item Всеобъемлющая сегментация(panoptic segmentation).
\end{itemize}

Таким образом какой-то конкретный алгоритм решает какую-то конкретную группу задач сегментации.

\subsection{Thing и stuff}

Чтобы объяснить разделение на группы введем два семантических класса: \textit{вещь(thing)} и \textit{вещество(stuff)}.

Вещь - это классы объектов с характерными формами, например, «автомобиль», «дерево» или «человек». Как правило, у вещей есть четко определенные экземпляры, которые можно подсчитать. Они имеют относительно небольшое изменение размера от одного экземпляра к другому, а также составные части, отличные от самой вещи: например, у всех автомобилей есть колеса, но колесо не является автомобилем.

Вещество относится к объектам, которые имеют аморфную форму и сильно варьируются по размеру, например, «небо», «вода» или «трава». Как правило, у вещества нет четко определенных, поддающихся подсчету отдельных экземпляров. В отличие от вещей, у вещества нет отдельных частей: одна травинка и трава, являются в одинаковой степени «травой».

Стоит отметить, что при определенных условиях изображения могут быть как «вещью», так и «веществом». Например, большая группа людей может быть интерпретирована как множество «людей», каждый из которых имеет определенную форму и поддается подсчету, или как единая, аморфная «толпа».

\subsection{Semantic segmentation}

Семантическая сегментация рассматривает все пиксели как «вещество», она не делает различий между «веществом» и «вещью». Таким образом, цель в том, чтобы разделить изображение на набор “веществ”. Такой подход не позволит выделить как объект машину, которую закрывает другая машина. Все будет рассмотрено как единый поток.

\subsection{Instance segmentation}

Сегментация экземпляров изменяет приоритеты семантической сегментации на противоположные: определение точной формы каждого отдельного экземпляра объекта, в то же время “вещество” она не рассматривает вовсе. Таким образом, сегментация экземпляров может быть рассмотрена, как усовершенствованная версия обнаружения объектов, первая обнаруживает точные очертания объектов, вторая – прямоугольники, в пределах которых находится объект или большая его часть.

Зачастую такая задача является более трудоемкой, чем семантическая сегментация. Это обусловлено тем, что каждый отдельный объект должен быть выделен, даже в случае наложения их друг на друга, в то время как семантическая сегментация может просто представить такой объект, как единое целое, вспомним пример с “людьми” и “толпой”.

\subsection{Panoptic segmentation}

Всеобъемлющая сегментация, нетрудно догадаться, сочетает преимущества как семантической сегментации, так и сегментации по экземплярам, определяя семантическую классификацию всех пикселей и разделяя каждый экземпляр объекта на изображении.

В задаче всеобъемлющей сегментации каждый пиксель должен быть помечен как семантической меткой, так и “идентификатором экземпляра”. Пиксели, имеющие одинаковую метку и идентификатор, принадлежат одному и тому же объекту; для пикселей, которые определены как материал, идентификатор экземпляра игнорируется.

Хотя привлекательность такого подхода очевидна, достижение всеобъемлющей сегментации последовательным и эффективным с точки зрения вычислений способом является крайне сложной задачей.

Задача заключается в объединении двух противоречащих друг другу методологий: модели семантической сегментации рассматривают все пиксели как материал, игнорируя отдельные экземпляры объектов; модели сегментации экземпляров выделяют отдельные объекты, игнорируя материал. Ни один из типов моделей не может адекватно выполнять функции другого.

Первоначальные попытки создания моделей всеобъемлющей сегментации просто объединили эти две модели, выполнив каждую задачу отдельно, а затем объединив их результаты на этапе постобработки. Этот подход имеет два основных недостатка: он требует больших вычислительных затрат и борется с расхождениями между точками данных, выводимыми сетью семантической сегментации, и точками данных, выводимыми сетью сегментации экземпляров.

Новые архитектуры всеобъемлющей сегментации направлены на то, чтобы избежать этих недостатков с помощью более унифицированного подхода к глубокому обучению. Большинство из них построены на основе “магистральной” сети, которая извлекает объекты из входного изображения, передает извлеченные данные в параллельные ветви — например, “ветвь переднего плана” и “ветвь фона”(семантическая и экземплярная части). Затем объединяет выходные данные каждой ветви, используя систему весов. Так получается распределение.

\subsection{Традиционный(классический) подход}
Традиционный или классический подход основывается на заранее заданных правилах(эвристиках) и математических принципах. Он полагается на анализ пикселей и их связей без использования обучающихся моделей.

Зачастую. чтобы получить вразумительный результат, пользователь должен вручную настраивать параметры, именуемые эвристиками. Они представляют собой пороговые значения, начальные точки кластеров и другие входные данные, которые зависят от конкретного алгоритма.

Стоит также отметить, что по сути не существует традиционных решений для всеобъемлющей сегментации, так как они могут быть лишь комбинацией нескольких традиционных методов. Забегая вперед, тоже самое можно сказать и про deep learning модели, которые берут за основу другие семантические и экземплярные модели, однако, если рассматривать их как черный ящик, то они берут на вход изображение и возвращают всеобъемлюще сегментированное изображение.

Приведу в пример некоторые традиционные алгоритмы:

\begin{itemize}
    \item \textbf{Пороговый метод(Thresholding)} разделяет пиксели изображения на два класса, сравнивая их интенсивность с заданным или автоматически определенным пороговым значением. Прокачивается с помощью метода Отсу, который дает возможность автоматически выбирать оптимальный порог, который минимизирует внутри классовую дисперсию и максимизирует межклассовую дисперсию.
    \item \textbf{Гистограммы}, отражающие частоту определенных значений пикселей на изображении, часто используются для определения пороговых значений, таким образом можно ставить не один порог, как в первом пункте, а несколько. Такой подход очень удобен, когда мы хотим отделить фон от объекта, однако фон не представляет собой единое “вещество”
    \item \textbf{Watersheds} – это сегментация, определяемая на изображении в оттенках серого. Название метафорически относится к геологическому водоразделу, или водоразделному водоразделу, который разделяет соседние водосборные бассейны. Преобразование водораздела обрабатывает изображение, с которым оно работает, как топографическую карту, где яркость каждой точки соответствует ее высоте, и находит линии, проходящие вдоль вершин хребтов
    \item \textbf{Clustering-based} алгоритмы основаны на группировке пикселей на основе их схожести. Смысл метода в том, чтобы объединить пиксели в группы (кластеры), где пиксели внутри одного кластера имеют схожие характеристики, такие как цвет, интенсивность или текстура
    \item \textbf{Edge detection} — это метод выделения контуров объектов на изображении путем выявления точек с резкими изменениями интенсивности пикселей. Этот процесс является основой для многих алгоритмов сегментации и анализа изображений.
\end{itemize}

\subsection{Deep learning подход}

Обученные на основе датасета нейронные сети моделей сегментации изображений с глубоким обучением обнаруживают лежащие в основе визуальных данных закономерности и выделяют характерные особенности, наиболее важные для классификации, обнаружения и сегментации.

Несмотря на различия в требованиях к вычислительной технике и времени обучения, модели глубокого обучения неизменно превосходят традиционные модели и формируют основу большинства текущих достижений в области компьютерного зрения.

Приведу пример нескольких известных deep learning моделей для сегментации изображений:

\begin{itemize}
    \item \textbf{Fully Convolutional Networks  (FCN)}: FCN, часто используемые для семантической сегментации, представляют собой тип сверточной нейронной сети (CNN) без фиксированных слоев. Сеть кодировщиков передает визуальные входные данные через сверточные слои для извлечения признаков, относящихся к сегментации или классификации, и сжимает (или сокращает выборку) эти данные признаков, чтобы удалить несущественную информацию. Затем эти сжатые данные передаются в слои декодера, где выполняется повышающая дискретизация извлеченных данных объектов для восстановления входного изображения с использованием масок сегментации.
    \item \textbf{Mask R-CNN} объединяют region proposal network (RPN), которая генерирует прямоугольные границы вокруг объектов с FCN, который представляет собой маску для этой границы. Таким образом, выполняется экземплярная сегментация
    \item \textbf{UPSNet} это современный подход к задаче всеобъемлющей сегментации, за семантическую часть в нем отвечает Feature Pyramid Network (FPN), а за экземплярную, описанный ранее Mask R-CNN, далее эти две ветки объединяются в одну, образуя всеобъемлющую сегментацию
\end{itemize}

\subsection{Заключение}

Существует еще огромное множество алгоритмов как традиционных, так и на основе глубокого обучения. Причем модели на основе deep learning появляются до сих пор. Я хотел лишь поведать о том, что такая простая задача для наших глаз, может стать настолько тяжелой задачей для машины, что либо ей приходится долгими неделями обучаться на датасетах, либо математикам придумывать алгоритмы для решения. Вот еще множество существующих решений для разных типов задач:

\begin{itemize}
    \item Алгоритмы, выполняющие сегментацию экземпляров: Selective Search, Edge Boxes, Mask R-CNN, YOLACT (You Only Look At Coefficients), BlendMask, SOLO (Segmenting Objects by Locations).
    \item Алгоритмы, выполняющие семантическую сегментацию: Watershed, Graph Cut, Random Walker, FCN (Fully Convolutional Networks), U-Net, DeepLab, PSPNet (Pyramid Scene Parsing Network).
    \item Алгоритмы, выполняющие всеобъемлющую сегментацию: Panoptic FPN, UPSNet, Panoptic-DeepLab.
\end{itemize}

\subsection{Источники}

\begin{itemize}
    \item \url{https://www.ibm.com/think/topics/image-segmentation}
    \item \url{https://se.mathworks.com/discovery/image-segmentation.html}
    \item \url{https://en.wikipedia.org/wiki/Image_segmentation}
    \item \url{https://en.wikipedia.org/wiki/Otsu%27s_method}
    \item \url{https://habr.com/ru/companies/intel/articles/266347/}
    \item \url{https://habr.com/ru/companies/data_light/articles/855336/}
    \item \url{https://scikit-image.org/docs/0.24.x/auto_examples/segmentation}
    \item \url{https://www.youtube.com/watch?v=gt5Fibr6uu0}
    \item \url{https://en.wikipedia.org/wiki/Watershed_(image_processing)}
    \item \url{https://paperswithcode.com/method/fpn}
    \item \url{https://paperswithcode.com/paper/upsnet-a-unified-panoptic-segmentation}
    \item \url{https://paperswithcode.com/paper/mask-r-cnn}
    \item \url{https://paperswithcode.com/paper/fully-convolutional-networks-for-semantic}
\end{itemize}
