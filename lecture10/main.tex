\section{Компьютерное зрение}
\begin{center}
    Конспект составил: \textit{Александр Рожков}
\end{center}

\subsection{Задачи компьютерного зрения}
\begin{itemize}
    \item Бинарная классификация: Im → 0, 1. Задача - определить наличие объекта на изображении (те дать конкретный ответ: да/нет).
    \item Многоклассовая классификация: Im → C \(\in\) \{1...n\}. Задача - присвоить изображению метку одного из n классов.
    \item Сегментация изображений: Im → Obj \(\in\) Im. Задача - выделить конкретный объект на изображении.
    \item Обнаружение объектов: Im → \{Obj, ...\}. Задача - найти несколько объектов на изображении.
    \item Разделение изображения на объекты: Im → \{Obj, ...\}, Im = \(\sum\) Obj. Задача - разбить изображение на составляющие его объекты (полное покрытие).
    \item Оценка «схожести» изображений: Сравнение изображений на предмет сходства.
    \item Определение «свойств» объектов: Описываем свойства уже выделенных объектов (цвет, размер и т.д.).
    \item Определение «свойств» изображения (общее): Описываем общие характеристики изображения: Im → tag, ... . Множество свойств может быть бесконечным.
\end{itemize}
\subsubsection{Признаки в компьютерной графике}
\begin{itemize}
    \item Цвета и их распределение.
    \item Геометрические признаки (формы, размеры, углы, расстояния).
    \item Градиент изображения (например, скорость изменения яркости).
    \item Структуры (имеется в виду линии у фигур, углы и тд).
    \item Соответствие образцу.
    \item Агрегация по нескольким изображениям (анализ видео).
\end{itemize}
\newpage
\subsection*{Пример}
Задача: Классификация изображений: Различение изображений кошек и
собак.

\*

\textbf{Извлекаем признаки:}

\begin{itemize}
    \item Цветовые признаки
\end{itemize}
Мы можем использовать гистограмму цветов. Те предположим, что
собаки и кошки имеют, в среднем, разный цвет шерстки. Поэтому
можно предположить, что если на фото преобладает больше серый,
то на фото кошка.
\begin{itemize}
    \item Геометрические признаки
\end{itemize}
Мы можем попытаться выделить форму морды, расположение глаз,
форму ушей и т.д. У собак и кошек они различаются.

\newpage
\subsection{Математическая статистика}
\begin{itemize}
    \item Математическое ожидание: Среднее значение признака.
    \item Дисперсия: Мера разброса значений относительно среднего.
\end{itemize}
\subsection*{Пример}
Анализ яркости области на изображении. Те представим, что у нас чернобелое изображение, которое состоит из пяти пикселей, яркость каждого из
которых можно описать значением от 0 (черный) до 255 (белый).
\begin{itemize}
    \item Картинка 1: 100, 110, 105, 95, 115
    \item Картинка 2: 50, 150, 25, 175, 100
\end{itemize}
\textbf{Математическое ожидание:}
\begin{itemize}
    \item Картинка 1: 105
    \item Картинка 2: 100
\end{itemize}
Математическое ожидание дало нам общее представление о яркости
каждой области
\textbf{Дисперсия:}
\begin{itemize}
    \item Картинка 1: 50
    \item Картинка 1: 3250
\end{itemize}
Дисперсия, в свою очередь, показала нам, насколько разнородны значения яркости внутри каждой области.

\newpage

\subsection{Гипотезы и проверка}
\begin{itemize}
    \item Основная гипотеза.

          Пример: "На изображении нет кошек."

    \item Альтернативная гипотеза.

          Пример: "На изображении есть кошка."

    \item Выборка: (X1, ...Xn) \(\sim\) F, где F – распределение, из которого получены данные. Данные независимы.
\end{itemize}
\subsection*{Ошибки при проверке гипотез}
\begin{itemize}
    \item Ошибка первого рода.
    \item Ошибка второго рода.
\end{itemize}
\subsection*{Пример}
Есть алгоритм, который должен определять, есть ли лица на фотографии.
\begin{itemize}
    \item \textbf{Нулевая гипотеза}: На фотографии нет лиц.
    \item \textbf{Альтернативная гипотеза}: На фотографии есть хотя бы одно лицо.
\end{itemize}

\*

\textbf{Ошибка первого рода}: Алгоритм анализирует фотографию, где на
самом деле нет лиц. Далее алгоритм \textbf{ошибочно отклоняет нулевую гипотезу} и сообщает, что на фотографии есть лицо. Из-за этого в галерею может добавиться помеченное фото с чьим-то лицом, что не плохо, но решаемо.

\*

\textbf{Ошибка второго рода}: Алгоритм анализирует фотографию, на которой действительно есть лица. Далее алгоритм \textbf{ошибочно принимает} \textbf{нулевую гипотезу} и не обнаруживает ни одного лица на фотографии. Из-за этого фотографии с людьми могут остаться не помеченными (семейный фотоальбом останется без фото, что невероятно грустно. \textbf{Эта ошибка страшнее}).